\documentclass[a4paper,11pt]{article}
\newcommand{\metres}{\ensuremath{\left[\,\mathrm{m\,}\right]}}
\newcommand{\seconds}{\ensuremath{\left[\,\mathrm{s}\,\right]}}
\newcommand{\kg}{\ensuremath{\left[\,\mathrm{kg}\,\right]}}
\usepackage{amsmath}
%\usepackage[margin=3cm]{geometry}
\setlength{\parskip}{6pt}
\setlength{\parindent}{0pt}
\usepackage{mathtools}  
\mathtoolsset{showonlyrefs} 
\title{Units for `pyprop8'}
\author{Andrew Valentine}
\begin{document}
\maketitle
The wave equation as stated in \cite{OToole2011} is
\begin{equation}
T_{ij,j} +f_i = \rho \partial_t^2 u_i
\end{equation}
with
\begin{align}
T_{ij} &= \mu(u_{i,j} + u_{j,i}) + \lambda u_{k,k} \delta_{ij}\\
\mu &=\rho v_s^2
\end{align}
and $f_i$ representing force \emph{per unit volume} (see \cite{Woodhouse2007})
\begin{equation}
f_i = (F_i - M_{ij}\partial_j)\delta(\mathbf{x}-\mathbf{x}_s)H(t)
\end{equation}
The purpose of this note is to derive the units of $u_i$ given the units of all other quantites.

We assume that all input quantities are specified in arbitrary units that can nevertheless be related back to the SI system:
\begin{align}
\mathrm{distance} &\to \alpha \metres\\
v_s, v_p &\to \alpha \metres \seconds^{-1}\\
\rho & \to \beta \kg \metres^{-3}\\
F_i & \to \gamma \kg \metres \seconds^{-2}\\
M_{ij} & \to \epsilon \kg \metres^2 \seconds^{-2}
\end{align}
For example, if velocities are expressed in $\mathrm{km}/\mathrm{s}$ we would have $\alpha = 10^3$. The resulting seismogram has units 
\begin{equation}
u_i \to \zeta \metres
\end{equation}
We are effectively seeking an expression for $\zeta$ in terms of $\alpha$, $\beta$, $\gamma$ and $\epsilon$.

We now determine the units of derived quantities. Straightforwardly, \[\lambda,\mu \to \alpha^2\beta\kg\metres^{-1}\seconds^{-2}\]
while $u_{i,j}$ requires us to recognise that the coordinate system is expressed in units of distance, so that \[u_{i,j}\to\frac{\zeta}{\alpha}\] Thus we have
\begin{align}
T_{ij} &\to \alpha\beta\zeta\kg\metres^{-1}\seconds^{-2}\\
T_{ij,j} &\to \beta\zeta\kg\metres^{-2}\seconds^{-2}
\end{align}
and straightforwardly
\begin{align}
\partial_t^2 u_i &\to \zeta \metres\seconds^{-2}\\
\rho \partial_t^2 u_i &\to \beta\zeta \kg\metres^{-2}\seconds^{-2}
\end{align}
showing that $T_{ij,j}$ and $\rho \partial_t^2 u_i$ are dimensionally consistent, as required.

For the force term, we have
\begin{equation}
M_{ij}\delta_j \to \frac{\epsilon}{\alpha}\kg \metres \seconds^{-2}
\end{equation}
and we must have 
\begin{equation}
\gamma = \frac{\epsilon}{\alpha}
\end{equation}
for dimensional consistency. Noting that the delta-function acts as a volumetric density distribution, we have
\begin{equation}
\delta(\mathbf{x}-\mathbf{x}_s) \to \frac{1}{\alpha^3}\metres^{-3}
\end{equation}
and thus 
\begin{equation}
f_i \to \frac{\epsilon}{\alpha^4} \kg \metres^{-2}\seconds^{-2}
\end{equation}
For this to be dimensionally-consistent with the other two terms, we require
\begin{equation}
\frac{\epsilon}{\alpha^4} = \beta \zeta
\end{equation}
Thus we find
\begin{equation}
\zeta = \frac{\epsilon}{\alpha^4\beta}
\end{equation}
and we have required $\gamma = \epsilon/\alpha$. We have also assumed that the spatial dimensions of the model, and the wave velocities, are expressed in the same system (i.e., $\alpha$ is shared by both).

If we express all model quantities in SI units (i.e. distances in metres, velocities in $\mathrm{m}/\mathrm{s}$, densities in $\mathrm{kg}/\mathrm{m^3}$, forces in $\mathrm{N}$ and moments in $\mathrm{N\,m}$, then all constants are 1 and output is also in SI units of metres. However doing so may be numerically inconvenient.

One convenient alternative, with `nice' numerical values for earth-like materials:
\begin{align}
\mathrm{distance}&\to \mathrm{km} &\implies \alpha &= 10^3\\
v_p, v_s &\to \mathrm{km}/\mathrm{s}\\
\rho & \to \mathrm{g}/\mathrm{cm}^{3} &\implies \beta &= 10^3
\end{align}
Thus $\zeta =10^{-15}\epsilon$. If we wish output to be in mm, which is convenient, then we require $\epsilon=10^{12}$ i.e. moment should be expressed in $\mathrm{TN\,m}$ (or, equivalently, $\mathrm{GN\,km}$). GCMT moment tensor values are quoted in units of $\mathrm{dyne\,cm}$, or equivalently $10^{-7} \mathrm{N\,m}$. We therefore need to multiply the numerical value of catalogue source parameters by a factor of $10^{-19}$ in order to achieve output in mm\footnote{Note that $x [\,\mathrm{units}\,] = \frac{x}{\alpha} [\alpha\,\mathrm{units}]$}. For completeness, we note that this system implies $\gamma = 10^9$ i.e. any force term should be expressed in GN.




\bibliographystyle{unsrt}
\begin{thebibliography}{1}

\bibitem{OToole2011}
T.B. O'Toole and J.H. Woodhouse.
\newblock Numerically stable computation of complete synthetic seismograms
  including the static displacement in plane layered media.
\newblock {\em Geophysical Journal International}, 187:1516--1536, 2011.

\bibitem{Woodhouse2007}
J.H. Woodhouse and A.F. Deuss.
\newblock {E}arth's free oscillations.
\newblock In B.~Romanowicz and A.~Dziewo\'{n}ski, editors, {\em Seismology and
  structure of the {E}arth}, volume~1 of {\em Treatise on Geophysics},
  chapter~2, pages 31--65. Elsevier, Amsterdam, 2007.

\end{thebibliography}



\end{document}